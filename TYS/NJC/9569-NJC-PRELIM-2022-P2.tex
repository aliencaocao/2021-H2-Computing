%% LyX 2.3.6.1 created this file.  For more info, see http://www.lyx.org/.
%% Do not edit unless you really know what you are doing.
\documentclass[11pt,twoside,english]{article}
\usepackage[T1]{fontenc}
\usepackage[latin9]{inputenc}
\usepackage{geometry}
\geometry{verbose,tmargin=2cm,bmargin=2cm,lmargin=2cm,rmargin=2cm}
\usepackage{graphicx}
\usepackage{setspace}

\makeatletter

%%%%%%%%%%%%%%%%%%%%%%%%%%%%%% LyX specific LaTeX commands.
%% Because html converters don't know tabularnewline
\providecommand{\tabularnewline}{\\}

%%%%%%%%%%%%%%%%%%%%%%%%%%%%%% User specified LaTeX commands.
\usepackage{helvet}
\renewcommand{\familydefault}{\sfdefault}
\usepackage[T1]{fontenc}
\usepackage[latin9]{inputenc}
\usepackage{geometry}
\geometry{verbose,tmargin=1.8cm,bmargin=4cm,lmargin=1.5cm,rmargin=2cm}
\usepackage{enumitem}
\usepackage{amstext}
\usepackage{amsthm}
\usepackage{amssymb}
\usepackage{setspace}
\usepackage{graphicx}
\doublespacing


\usepackage{enumitem}
\setenumerate[1]{label=\textbf{\arabic*}}
\setenumerate[2]{label=\textbf{(\alph*)}}
\setenumerate[3]{label=\textbf{(\roman*)}}
\setlist[enumerate]{align=right}

\setcounter{page}{2}

%for upright integrals
\usepackage[integrals]{wasysym}

%to be used in conjunction with fancyfoot for last page
\usepackage{zref-totpages}

%fancyhrd settings
\usepackage{fancyhdr}
\pagestyle{fancy}
\fancyhf{}

\fancypagestyle{laststyle}
{
   \fancyhf{}
   \chead{\thepage}
   \fancyfoot[L]{\copyright NJC }
   \fancyfoot[R]{\textbf{END}} %Put \thispagestyle{laststyle} in the last page
}

%%centering page number
\chead{\thepage}

\renewcommand{\headrulewidth}{0pt}
\renewcommand{\footrulewidth}{0pt}

%%footer settings, different footer for ODD and EVEN pages, also for the LASTPAGE
\fancyfoot[LO]{\copyright NJC\hfill \textbf{[Turn Over}}
\fancyfoot[LE]{\copyright NJC }

%%shameless self-plug BRW

\makeatother

\usepackage{babel}
\begin{document}
\begin{enumerate}
\item In mathematics, a polynomial in $x$ is an expression $p\left(x\right)$
consisting of a single variable $x$ and coefficients, that involves
only the operations of addition, subtraction, multiplication, and
non-negative integer exponentiation of $x$.

For example, $x^{2}-4x+7$, $8x^{3}-5x+6$ are examples of polynomials.

\subsection*{Task 1.1}

The Python function \texttt{eval()} can be used to evaluate a mathematical
expression in the string format to give a numerical value. For example,
\texttt{eval('5{*}{*}2+3{*}4-1')} will return the value \texttt{36}. 

Write a function \texttt{fun(expression,num) }that takes in a string
expression of a polynomial and evaluate the value of the polynomial
when the variable $x$ in the polynomial is set to \texttt{num. }\hfill{}{[}2{]}

Polynomials can be plotted on the $xy$-axes and the area of the region
$R$ bounded by the polynomial $p\left(x\right)$, the lines $x=a$
, $x=b$, and the $x$-axes is a quantity of interest to a lot of
people. For this question, we assume $a\le b$.

\newpage

\subsection*{Task 1.2}

One way to obtain the approximation of the area of the region $R$
is to do the following procedure:
\begin{enumerate}
\item divide the line segment between the vertical lines $x=a$ and $x=b$
into $n$ segment of equal lengths $\Delta x$,
\item for each line segment produced, we generate a rectangle with width
$\Delta x$ and the height is the value of the polynomial at the \textbf{left}most
$x$-value of the line segment,
\item compute the area of the rectangles generated this way and sum them,
\item The sum is the approximated area of the region $R$.
\end{enumerate}
The following is an example of the approximation given by the procedure
above, when $a=2,$$b=4$, $n=2$ and polynomial $p\left(x\right)=x^{2}-4x+7$. 
\noindent \begin{center}
\includegraphics[scale=0.4]{TASK1_1}
\par\end{center}

Write a function \texttt{left\_sum(expr, a, b, num) }to implement
the process above. The function in a polynomial expression \texttt{expr},
and numerics \texttt{a}, \texttt{b} and \texttt{num}, where \texttt{a}
and \texttt{b} are the $x$-values of the region $R$ and \texttt{num
}is the number of line segments we are using to do the approximation.\hfill{}{[}3{]}

Test your function with the following call:

\texttt{left\_sum('8{*}x{*}{*}3-5{*}x+6', 2, 4, 100)}\hfill{}{[}1{]}

\newpage

\subsection*{Task 1.3}

Yet another way to obtain the approximation of the area of the region
$R$ is to use a similar procedure as of Task 1.2, with the following
modification to part (a) and (b):
\begin{enumerate}
\item[\textbf{(a'')}] divide the line segment between the vertical lines $x=a$ and $x=b$
into $n$ line segments $l_{i}$ of random lengths. The \texttt{random.uniform(a,b)}
method generates a random value between $a$ and $b$.
\item[\textbf{(b'')}] for each line segment $l_{i}$ produced, we generate a rectangle
with width $\Delta x_{i}$ and the height is the value of the polynomial
at middle $x$-value of the line segment,
\end{enumerate}
The following is an example of the approximation given by the procedure
above, when $a=2,$$b=4$, $n=2$, polynomial $p\left(x\right)=x^{2}-4x+7$.
The two intervals $l_{1}$ and $l_{2}$ in the example has lengths
$\Delta x_{1}=1.2$ units and $\Delta x_{2}=0.8$ units.
\noindent \begin{center}
\includegraphics[scale=0.4]{TASK1_3}
\par\end{center}

Write a function \texttt{random\_sum(expr, a, b, num) }to implement
the process above. The function in a polynomial expression \texttt{expr},
and numerics \texttt{a}, \texttt{b} and \texttt{num}, where \texttt{a}
and \texttt{b} are the $x$-values of the region $R$ and \texttt{num
}is the number of line segments we are using to do the approximation.\hfill{}{[}4{]}

Test your function with the following call:

\texttt{random\_sum('8{*}x{*}{*}3-5{*}x+6', 2, 4, 100)}\hfill{}{[}1{]}

\newpage
\item For this question you are provided with three text files, each contains
a valid list of positive integers, one per line:
\begin{itemize}
\item \texttt{TEN.txt} has 10 lines
\item \texttt{HUNDRED.txt} has 100 lines
\item \texttt{THOUSAND.txt} has 1000 lines.
\end{itemize}
For each of the sub-tasks, add a comment statement at the beginning
of the code using the hash symbol \textquoteleft \#' to indicate the
sub-task the program code belongs to, for example:

\begin{singlespace}
\noindent \texttt{}%
\begin{tabular}{c|lcccccccccccccccccccccc|}
\cline{2-24} \cline{3-24} \cline{4-24} \cline{5-24} \cline{6-24} \cline{7-24} \cline{8-24} \cline{9-24} \cline{10-24} \cline{11-24} \cline{12-24} \cline{13-24} \cline{14-24} \cline{15-24} \cline{16-24} \cline{17-24} \cline{18-24} \cline{19-24} \cline{20-24} \cline{21-24} \cline{22-24} \cline{23-24} \cline{24-24} 
\texttt{In {[}1{]} :} & \texttt{\#Task 2.1} &  &  &  &  &  &  &  &  &  &  &  &  &  &  &  &  &  &  &  &  &  & \tabularnewline
 & \texttt{Program Code} &  &  &  &  &  &  &  &  &  &  &  &  &  &  &  &  &  &  &  &  &  & \tabularnewline
\cline{2-24} \cline{3-24} \cline{4-24} \cline{5-24} \cline{6-24} \cline{7-24} \cline{8-24} \cline{9-24} \cline{10-24} \cline{11-24} \cline{12-24} \cline{13-24} \cline{14-24} \cline{15-24} \cline{16-24} \cline{17-24} \cline{18-24} \cline{19-24} \cline{20-24} \cline{21-24} \cline{22-24} \cline{23-24} \cline{24-24} 
\multicolumn{1}{c}{} & \texttt{Output:} &  &  &  &  &  &  &  &  &  &  &  &  &  &  &  &  &  &  &  &  &  & \multicolumn{1}{c}{}\tabularnewline
\end{tabular}
\end{singlespace}

\subsection*{Task 2.1 }

Write a function \texttt{task2\_1(filename)} that: 
\begin{itemize}
\item takes a string filename which represents the name of a text file 
\item reads in the contents of the text file 
\item returns the content as a list of integers. \hfill{}{[}3{]} 
\end{itemize}

\subsection*{Task 2.2 }

One method of sorting is the bubble sort.

Write a function \texttt{task2\_2(list\_of\_integers)} that: 
\begin{itemize}
\item takes a list of integers 
\item implements a bubble sort algorithm 
\item returns the sorted list of integers in ascending order. \hfill{}{[}5{]}
\end{itemize}
\newpage

\subsection*{Task 2.3 }

Another method of sorting is the insertion sort. 

Write a function t\texttt{ask2\_3(list\_of\_integers)} that: 
\begin{itemize}
\item takes a list of integers 
\item implements a insertion sort algorithm 
\item returns the sorted list of integers in ascending order.\hfill{} {[}5{]}
\end{itemize}

\subsection*{Task 2.4 }

Yet another method of sorting is the quicksort. 

Write a function t\texttt{ask2\_4(list\_of\_integers)} that: 
\begin{itemize}
\item takes a list of integers 
\item implements a quicksort algorithm 
\item returns the sorted list of integers in ascending order.\hfill{} {[}7{]}
\end{itemize}

\subsection*{Task 2.5 }

Even another method of sorting is the merge sort. 

Write a function t\texttt{ask2\_5(list\_of\_integers)} that: 
\begin{itemize}
\item takes a list of integers 
\item implements a merge sort algorithm 
\item returns the sorted list of integers in ascending order.\hfill{} {[}7{]}
\end{itemize}
\newpage

\subsection*{Task 2.6}

Call your functions \texttt{task2\_2 task2\_3, task2\_4, task2\_5}
with the contents of the file \texttt{TEN.txt}, printing the returned
lists. For example, using the following statement: 

\texttt{\qquad{}print(task2\_2(task2\_1('TEN.txt')))}.\hfill{}{[}1{]}

\subsection*{Task 2.7}

The \texttt{timeit} library is built into Python and can be used to
time simple function calls. Example code is shown in \texttt{Task2\_timing.py}.
(The sample code assumes that it has access \texttt{task2\_2} function.) 

Using the \texttt{timeit} module, or other evidence, and the three
text files provided with this question, compare and contrast, including
mention of orders of growth, the time complexity of the different
sorting algorithms. 

Save your Jupyter notebook for Task 2.\hfill{}{[}5{]}

\newpage
\item A hobbyist programmer is trying to create a web browser. A feature
he wants the web browser to have is the ability to revisit previously
accessed web pages. To achieve this, he designed a class, \texttt{Stack},
to represent a stack of the user's browsing history, which includes
the web pages with its associated page title and time of visit. 

For each of the sub-tasks, add a comment statement at the beginning
of the code using the hash symbol \textquoteleft \#' to indicate the
sub-task the program code belongs to, for example:

\begin{singlespace}
\noindent \texttt{}%
\begin{tabular}{c|lcccccccccccccccccccccc|}
\cline{2-24} \cline{3-24} \cline{4-24} \cline{5-24} \cline{6-24} \cline{7-24} \cline{8-24} \cline{9-24} \cline{10-24} \cline{11-24} \cline{12-24} \cline{13-24} \cline{14-24} \cline{15-24} \cline{16-24} \cline{17-24} \cline{18-24} \cline{19-24} \cline{20-24} \cline{21-24} \cline{22-24} \cline{23-24} \cline{24-24} 
\texttt{In {[}1{]} :} & \texttt{\#Task 3.1} &  &  &  &  &  &  &  &  &  &  &  &  &  &  &  &  &  &  &  &  &  & \tabularnewline
 & \texttt{Program Code} &  &  &  &  &  &  &  &  &  &  &  &  &  &  &  &  &  &  &  &  &  & \tabularnewline
\cline{2-24} \cline{3-24} \cline{4-24} \cline{5-24} \cline{6-24} \cline{7-24} \cline{8-24} \cline{9-24} \cline{10-24} \cline{11-24} \cline{12-24} \cline{13-24} \cline{14-24} \cline{15-24} \cline{16-24} \cline{17-24} \cline{18-24} \cline{19-24} \cline{20-24} \cline{21-24} \cline{22-24} \cline{23-24} \cline{24-24} 
\multicolumn{1}{c}{} & \texttt{Output:} &  &  &  &  &  &  &  &  &  &  &  &  &  &  &  &  &  &  &  &  &  & \multicolumn{1}{c}{}\tabularnewline
\end{tabular}
\end{singlespace}

\subsection*{Task 3.1 }

To facilitate the ease of retrieval of the pages, the programmer decide
to create a class \texttt{Page }to store the addresses, its title
and the timestamp when it was first accessed. The class will store
the following data:
\begin{itemize}
\item \texttt{page\_address} - stored as a string
\item \texttt{page\_title} - also stored as a string
\item \texttt{time\_stamp} - a string with the format \texttt{YYYY-MM-DDTHH:MM:SSZ},
where \texttt{YYYY-MM-DD }and\texttt{ HH:MM:SS }are the dates and
hours at which the addresses were visited.
\end{itemize}
The class has the following methods defined on it:
\begin{itemize}
\item \texttt{get\_address()}: returns the address of the visited page
\item \texttt{get\_title()}: returns the title of the visited page
\item \texttt{get\_date()}: returns \texttt{YYYY-MM-DD }part of the time
stamp
\item \texttt{get\_hour()}: returns \texttt{HH:MM:SS }part of the time stamp
\item \texttt{display()}: returns a string of the form \texttt{Page(<page\_address>,
<page\_title>, <time\_stamp>) }where \texttt{<page\_address>}, \texttt{<page\_title>},
\texttt{<time\_stamp> }are the corresponding attributes of the \texttt{Page}
object.
\end{itemize}
Write the class \texttt{Page }with its associated methods in Python.
\hfill{}{[}5{]}

\newpage

\subsection*{Task 3.2 }

Write the \texttt{Stack} class in Python. Use of a simple Python list
is not sufficient. Include the following methods: 
\begin{itemize}
\item \texttt{push(page\_object)} inserts the \texttt{page\_object} at the
top of the stack 
\item \texttt{pop()} attempts to pop the \texttt{page\_object} at the top
of the stack; if the item was not present, return \texttt{None} 
\item \texttt{search(page\_title)} returns a Boolean value: \texttt{True}
if\texttt{ }a page with title \texttt{page\_title} is in the stack,
\texttt{False} if not in the stack 
\item \texttt{count()} should return the number of elements in the stack,
or zero if empty 
\item \texttt{to\_String()} should return a string containing a suitably
formatted list with the elements being page objects in the stack,
separated by a comma and a space, with square brackets at either end,
eg. in the form: 

\begin{minipage}[t]{0.8\columnwidth}%
\texttt{{[}}

\texttt{Page(https://ovh.net/diam/erat/fermentum/justo.html, Monitored
radical synergy, 2021-09-17T09:33:05Z),}

\texttt{Page(http://bloglovin.com/felis/donec/semper.xml, De-engineered
asymmetric structure, 2021-09-17T09:33:35Z)}

\texttt{{]}}%
\end{minipage}

\hfill{}{[}9{]}
\end{itemize}
Test \texttt{Stack} by using the data in the file \texttt{Task3data1.csv},
where line 1 is the first site visited. Use the \texttt{to\_String()}
method to print the contents of the stack. \hfill{}{[}3{]}

\newpage

\subsection*{Task 3.3}

To manage the number of browsing history, the programmer decided to
limit the number of elements in the stack to 12.

Write a Python subclass \texttt{ModStack} using \texttt{Stack} as
its superclass. 

The \texttt{push} method in the \texttt{ModStack} subclass should
ensure if another page is accessed when the number of elements in
the new data structure is 12, the earliest visited address will be
removed from the history and the latest address is pushed to the top
of the new data structure. \hfill{}{[}4{]}

Test \texttt{ModStack} by using the data in the file \texttt{Task3data2.csv},
where line 1 is the first site visited. Print the result of searching
the \texttt{ModStack} for the page with a title \texttt{'Object-based
global firmware'}. \hfill{}{[}2{]}

\subsection*{Task 3.4 }

After some experimentation, the programmer realised that using a stack
like data structure for a browsing history is rather silly as the
user can only move to latest previously visited page before returning
back to the desired page. And, once he reached the desired page, the
data structure no longer holds the browsing history after that page. 

As such, he decided to use a linked list instead. 

For this task, ignore the browsing history limit mentioned in Task
3.3.

Write a Python subclass \texttt{PageNode} using \texttt{Page} as its
superclass to be used in the linked list. Include the following attributes
for the PageNode object:
\begin{itemize}
\item \texttt{next} : initializes to \texttt{None}. Stores a pointer to
the next \texttt{PageNode} object in the linked list after the current
\texttt{PageNode} object.
\end{itemize}
The subclass also has the following methods defined on it:
\begin{itemize}
\item \texttt{get\_next()}: returns the next \texttt{PageNode} object in
the linked list. If there's no such object, return \texttt{None}.
\item \texttt{set\_next(page)}:\texttt{ }set the pointer from the node to
the next \texttt{PageNode} object in the linked list. 
\end{itemize}
Write the \texttt{LinkedList} class in Python. Use of a simple Python
list is not sufficient. Include the following methods: 
\begin{itemize}
\item \texttt{insert(page\_object)} inserts the \texttt{page\_object} at
the end of the list 
\item \texttt{delete(page\_title)} attempts to delete a \texttt{PageNode
}object containing data \texttt{page\_title} from the list; if the
item was not present, return \texttt{None.}
\item \texttt{to\_String()} should return a string containing a suitably
formatted list with the elements separated by a comma and a space,
with square brackets at either end, eg. in the form: 

\begin{minipage}[t]{0.8\columnwidth}%
\texttt{{[}}

\texttt{Page(https://ovh.net/diam/erat/fermentum/justo.html, Monitored
radical synergy, 2021-09-17T09:33:05Z),}

\texttt{Page(http://bloglovin.com/felis/donec/semper.xml, De-engineered
asymmetric structure, 2021-09-17T09:33:35Z)}

\texttt{{]}}%
\end{minipage}
\item \texttt{visit(page\_title) }is a procedure that emulates the user
visiting an address with a given page title \texttt{page\_title} in
the history\texttt{.} The page title will be searched from the head
of the linked list. If the page title is found, the node will first
be removed from the linked list and then placed back at the tail of
the list. If the page is not found, print \texttt{'Page Not Found
in History'}.
\end{itemize}
\hfill{}{[}9{]}

Create two \texttt{LinkedList} objects by using the data in the file
\texttt{Task3data2.csv, }where line 1 is the first site visited.

Test the objects with the following methods:
\begin{itemize}
\item \texttt{visit('Streamlined didactic matrix')},
\item \texttt{visit('Advanced locale concept').}
\end{itemize}
Use the \texttt{to\_String()} method to print the resulting contents
of the list.\hfill{}{[}2{]}

Save your Jupyter notebook for Task 3.

\newpage
\item A text file, \texttt{TIDES.TXT}. contains the low and high tide information
for a coastal location tor each day of a month. Each line contains
tab-delimited data that shows the date, the time. whether the tide
is high or low and the tide height in metres. 

Each line is in the format: 
\begin{center}
\texttt{YYYY-MM-DD\textbackslash tHH:mm\textbackslash tTIDE\textbackslash tHEIGHT\textbackslash n }
\par\end{center}
\begin{itemize}
\item The date is in the term YYYY-MM-DD, for example. 2019-08-03 is 3rd
August. 2019 
\item The time is in the form HH:mm, for example. 13:47 
\item TIDE is either HIGH or LOW 
\item HEIGHT is a positive number shown to one decimal place 
\item \texttt{\textbackslash t} represents the tab character
\item \texttt{\textbackslash n} represents the newline character 
\end{itemize}
The text file is stored in ascending order of date and time.

\subsubsection*{Task 4.1}

Write program code to:
\begin{itemize}
\item read the tide data from the text file \texttt{TIDES.TXT}
\item insert all information from the file into a SQLite database called
\texttt{tide.db} with the single table, \texttt{Tide}. The table will
have the following fields of the given SQLite types:
\begin{itemize}
\item \texttt{RecordID} - primary key, an auto-incremented integer
\item \texttt{Date} - the date in the format YYYY-MM-DD, text
\item \texttt{Time} - the time in the format HH:mm, text
\item \texttt{isHigh} - a Boolean using 0 if the field entry for the row
is \texttt{LOW} and 1 if the entry is \texttt{HIGH}, integer
\item \texttt{Height }- positive number shown to one decimal place, real
\hfill{}{[}4{]}
\end{itemize}
\end{itemize}
\newpage

\subsubsection*{Task 4.2}

A researcher decided that he wanted to use the information in \texttt{tide.db}
to calculate some values from a remote computer. As such, he decided
to use socket programming to achieve his goal. 

Write a server program that:
\begin{itemize}
\item instantiates a server socket,
\item binds the socket to localhost and port number 8888,
\item listens for incoming request, accepts incoming request and establishes
connection with the client,
\item presents the client with a menu with the following options:

\begin{tabular}{|ll|}
\hline 
1. & Highest high tide with one corresponding date and time it happened\tabularnewline
2. & Lowest low tide with one corresponding date and time it happened\tabularnewline
3. & Largest tidal range\tabularnewline
4. & Smallest tidal range\tabularnewline
\hline 
\end{tabular}
\item calculates the values (round to 1 decimal places) requested by the
client from the menu by accessing \texttt{tide.db}. The \emph{tidal
range} in the menu is defined to be the absolute difference between
the heights of successive tides; from a high tide to the following
low tide or from a low tide to the following high tide, 
\item send the requested value to the client.
\end{itemize}
Save your server program in the file \texttt{TASK4\_SERVER\_<your
name>.py.}\hfill{}{[}9{]}

\subsubsection*{Task 4.3}

Write a corresponding client program that can:
\begin{itemize}
\item connect to the server,
\item asks the user for an input to request for the values that the server
can calculate,
\item close the sockets when the service from server is no longer required.
\end{itemize}
Save your client program in the file \texttt{TASK4\_CLIENT\_<your
name>.py.}\hfill{}{[}3{]}

\thispagestyle{laststyle}
\end{enumerate}

\end{document}
